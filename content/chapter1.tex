% content/chapter1.tex
\section{The Geometric Interpretation of Systems of Equations}

    \subsection{Two equations, two unknowns}
        \indent We have the following system of equations:
            \begin{subequations} \label{eq:ex1}
               \begin{empheq}[left=\empheqlbrace]{align}
                    2x-y &= 0 \label{eq:ex1_a} \\
                    -x + 2y &= 3 \label{eq:ex1_b}
                \end{empheq}
            \end{subequations}
             
        
        First, let's observe the system from the perspective of a matrix.
        From the system,we can extract the coefficient matrix:
            \begin{equation}
                \mathbf{A} = 
                \begin{bmatrix}
                    2 & -1 \\
                    -1 & 2
                \end{bmatrix} \label{eq:ex1_coefficient}
            \end{equation}
        the unknown vector:
            \begin{equation}
                \mathbf{x} = 
                \begin{bmatrix}
                    x \\
                    y
                \end{bmatrix} \label{eq:ex1_unknown}
            \end{equation}
        and the right vector is:
            \begin{equation}
                \mathbf{b} = 
                \begin{bmatrix}
                    0 \\
                    3
                \end{bmatrix} \label{eq:ex1_right}
            \end{equation}
        so the system of equations can be written in matrix form:
            \begin{equation}
                \mathbf{A} \mathbf{x} = \mathbf{b}
            \end{equation}

        Second, we give the \textbf{row picture}. We can find that, 
        each row of \eqref{eq:ex1} corresponds to a staright line. 
        Draw it on the rectangular coordinate system of a plane:
            \inserttikzpicture
                { % 第一个参数: TikZ 绘图代码
                    % 1. 坐标轴范围
                    \def\xmin{-3}
                    \def\xmax{3}
                    \def\ymin{-3}
                    \def\ymax{3}
                    
                    % 添加刻度
                    \draw[->, thick] (\xmin, 0) -- (\xmax, 0) node[below] {$x$};
                    \draw[->, thick] (0, \ymin) -- (0, \ymax) node[left] {$y$};
                    \node[below left, font=\tiny] at (0,0) {$O$};
                    \foreach \x in {-2,-1,1,2} {
                        \draw (\x, 0.1) -- (\x, -0.1) node[below] {$\x$};
                    }
                    \foreach \y in {-2,-1,1,2} {
                        \draw (0.1, \y) -- (-0.1, \y) node[left] {$\y$};
                    }
                    % 添加网格
                    \draw[gray!50, dashed, very thin] (\xmin, \ymin) grid (\xmax, \ymax);
                    % 裁剪区域
                    \clip (\xmin, \ymin) rectangle (\xmax, \ymax);                 
                    % 绘制直线 
                    \draw[domain=\xmin:\xmax, color=blue, thick, samples=100]
                        plot (\x, {2*\x});
                    \draw[domain=\xmin:\xmax, color=red, thick, samples=100]
                        plot (\x, {0.5*\x + 1.5});
                    % 使用绝对坐标放置标签
                    \node[blue,right] at (0.3, 0.5) {$2x-y=0$};
                    \node[red, right] at (-2, 0.5) {$-x+2y=3$};
                    % 标记交点 (依然使用绝对坐标)
                    \filldraw[black] (1,2) circle (2pt) node[right] at (1,2) {$(1, 2)$};
                }
                {Two lines intersect} % 第二个参数: 图片标题
                {fig:two_lines_intersect} % 第三个参数: 图片标签
        \noindent The intersection point $(1,2)$ is the solution of \eqref{eq:ex1}.

        Finally, we give the \textbf{column picture}.We can write the system in the below form:
            \begin{equation}
                x
                \begin{bmatrix}
                    2\\
                    -1
                \end{bmatrix} + 
                y
                \begin{bmatrix}
                    -1\\
                    2
                \end{bmatrix} = 
                \begin{bmatrix}
                    0\\
                    3
                \end{bmatrix} \label{eq:ex1_column}
            \end{equation}
        We can understand the above formula as the \textbf{linear combination of column vectors}:
            \inserttikzpicture
                { % 第一个参数: TikZ 绘图代码
                    % 1.坐标轴范围及定义
                    \def\xmin{-1}
                    \def\xmax{3}
                    \def\ymin{-1}
                    \def\ymax{3}
                    % 绘制坐标轴
                    \draw[->, thick, -{Stealth[length=2.5mm]}] (\xmin, 0) -- (\xmax, 0) node[below] {$x$};
                    \draw[->, thick, -{Stealth[length=2.5mm]}] (0, \ymin) -- (0, \ymax) node[left] {$y$};
                    \node[below left, font=\tiny] at (0,0) {$O$}; % 标记原点
                    % 3. 网格和刻度
                    \draw[gray!30, dashed, thin] (\xmin, \ymin) grid (\xmax, \ymax);
                    \foreach \x in {\xmin,...,\xmax} {
                        \ifnum\x=0\relax\else % 跳过原点
                            \ifnum\x<\xmax\relax
                                \draw (\x, 0.1) -- (\x, -0.1) node[below] {$\x$};
                            \fi
                        \fi
                    }
                    \foreach \y in {\ymin,...,\ymax} {
                        \ifnum\y=0\relax\else % 跳过原点
                            \ifnum\y<\ymax\relax
                                \draw (0.1, \y) -- (-0.1, \y) node[left] {$\y$};
                            \fi
                        \fi
                    }
                    % 4.绘制三个箭头向量
                    % 向量 a: (-1, 2) 
                    \draw[->, black, thick, -{Stealth[length=3mm, width=2mm]}] (0,0) -- (-1,2) node[above, black] {$(-1,2)$};
                    % 向量 b: (0, 3) 
                    \draw[->, red, thick, -{Stealth[length=3mm, width=2mm]}] (0,0) -- (0,3) node[right, red] {$(0,3)$};
                    % 向量 c: (2, -1) 
                    \draw[->, black, thick, -{Stealth[length=3mm, width=2mm]}] (0,0) -- (2,-1) node[right, black] {$(2,-1)$};
                }
                {Column Picture} % 第二个参数: 图片标题
                {fig:three_vectors} % 第三个参数: 图片标签
        
    \subsection{A simple summary}
        In the previous subsection, we use matrix form, row picture, column picture to describe
        the system of equations. And the most important point is the column picture.

        The column picture tells us that, we can use the linear combination of two vectors (satisfied certain conditions)
        to obtain the third vector.

        \textbf{Note}:If we take x and y as any combination of numbers,
        than the linear combination of tow vectors (satisfied certain conditions) can obtain every vectors in the plane. 
    
    \subsection{Three equations,three unknowns}
        We have the following system of equations:
            \begin{subequations} \label{eq:ex2}
               \begin{empheq}[left=\empheqlbrace]{align}
                    2x-y &= 0  \\
                    -x + 2y -z &= 3 \\
                    -3y + 4z &= 4
                \end{empheq}
            \end{subequations}
        First, let's give the matrix form. The coefficient matrix is:
            \begin{equation} \label{eq:ex2_A}
                \mathbf{A} = 
                \begin{bmatrix}
                    2 & -1 & 0 \\
                    -1 & 2 & -1 \\
                    0 & -3 & 4 
                \end{bmatrix}
            \end{equation}
        the unkown vector:
            \begin{equation}
                \mathbf{x} = 
                \begin{bmatrix}
                    x \\
                    y \\
                    z 
                \end{bmatrix}
            \end{equation}
        and the right vector is:
            \begin{equation}
                \mathbf{b} = 
                \begin{bmatrix}
                    0 \\
                    -1 \\
                    4 
                \end{bmatrix}
            \end{equation}
        so the system can be written:
            \begin{equation}
                \mathbf{Ax = b}
            \end{equation}
        
        Second, the row picture. With the aid of the geometric knowledge, we know that,
        every row of \eqref{eq:ex2} corresponds to a plane in 3-D space.

        Finally, the column picture. Below:
            \begin{equation} \label{eq:ex2_column}
                x
                \begin{bmatrix}
                    2 \\ 
                    -1\\
                    3
                \end{bmatrix} + 
                y
                \begin{bmatrix}
                    -1 \\ 
                    2\\
                    -3
                \end{bmatrix} + 
                z
                \begin{bmatrix}
                    0 \\ 
                    -1\\
                    4
                \end{bmatrix} = 
                \begin{bmatrix}
                    0 \\ 
                    -1\\
                    4
                \end{bmatrix} 
            \end{equation}
        As x, y and z are scalars, so the left-side of \eqref{eq:ex2_column} is the linear combination of three vectors.
        And easily, the solution is:
            \begin{equation}
                \begin{bmatrix}
                    0\\
                    0\\
                    1
                \end{bmatrix}
            \end{equation}
        
    \subsection{Extended thinking}
        In the previous subsection, we get the solution of \eqref{eq:ex2}.

        So, for the fixed A \eqref{eq:ex2_A}, can I solve $\mathbf{Ax=b}$ for any $\mathbf{b}$?
        Or, in the linear combination words, the problem is: Using any combination of ($x,y,z$), 
        can I probuce any vectors in a 3-D space?
        
        For A\eqref{eq:ex2_A}, the answer is YES. Then in what case, the answer is NO? 

        We can give an example: Three vectors in $\mathbf{A}$ are in the same plane. 
        In this case, we could't get any vector outside the plane.
        